In this work, we have demonstrated the optimization process of a complete RRI program using polyhedral code generation tool. We have explored different schedules, memory maps, and tiling transformations for our optimization work using the polyhedral code generator -  \alphaz\ .

We have explored multi-level tiling in our optimization work. We observe that a register-tiled kernel was easier to integrate when the program was transformed using mono-parametric tiling. Also, mono-parametric tiling enabled us to tile the nearly tile-able OSP-like inner-reductions ($R^{1}$ and $R^{2}$). We achieved more than $50\%$ of the roofline machine peak and improved the performance of the entire BPMax program by $400\times$. $70\%$ of this work got done by the register-tiled loop. Analysis from Intel Advisor shows that this loop attained $80\%$ of the roofline machine peak. In the future, it will be interesting to understand if there are further optimization opportunities like additional memory transformation to get close to the roofline machine peak.

We observed that double max-plus operation on single-core attained $80\%$ of the roofline machine peak. But, performance dropped to $53\%$ of the roofline machine peak with eight threads. So, finding opportunities to mitigate the scheduling and communication latency between the threads will be interesting. Our optimization work focused on the Intel platform. However, a future direction will be implementing the register-tiled kernel for a different processor architecture like AMD and comparing the performance improvement. In the long term, it can also be beneficial to distribute the computation over a cluster using MPI (Message Passing Interface) program to take advantage of another level of parallelism. All these transformations remain a challenge for \alphaz\ today. So, we also envision future work on \alphaz\ to allow these advanced transformations.

%Finally, we expect similar polyhedral transformations to be easily applied to more complex RRI algorithms like BPPart and piRNA using \alphaz\ to generate optimized code and achieve significant speedup.


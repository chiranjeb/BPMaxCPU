\subsection{Previous BPMax Schedules}
\textbf{Original BPMax Schedule:}
Let us recall that the original BPMax computed a four-dimensional table $F$-table based on five reductions ($R^{0}$, $R^{1}$, $R^{2}$, $R^{3}$, $R^{4}$) and two two-dimensional tables - $S^{(1)}$ and $S^{(2)}$. \alphaz\ treats each of these entities as a unique variable and requires the user to specify a schedule and a memory map. We have observed previously that $S^{(1)}$ and $S^{(2)}$ require only the input sequences. Thus, they can be scheduled before any other reductions. The schedules of the remaining variables are formulated based on the wavefront parallelization of the 6-D schedule space. We highlight the original program schedule in Table~\ref{tab:bpm_original_schedule}.
\begin{table}[htbp]
\caption{\uppercase{BPMax original Parallelization}}
\begin{center}
\begin{tabular}{|c|c|}
\hline
\textbf{\textit{Reduction}}& \textbf{\textit{Schedules}} \\
\hline
$S^{(1)}$ & $(i_{1},j_{1}, k_{1} \mapsto  0, j_{1}-i_{1}, i_{1}, k_{1}, 0, 0, 0)$   \\
\cline{1-2} 
$S^{(2)}$ & $(i_{2},j_{2}, k_{2} \mapsto  0, j_{2}-i_{2}, i_{2}, k_{2}, 1, 1, 1)$   \\
\cline{1-2} 
$R^{0}$ & $(i_{1},j_{1},i_{2},j_{2},k_{1},k_{2} \mapsto 1, j_{1}-i_{1}, j_{2}-i_{2}, i_{1}, i_{2}, k_{1}, k_{2})^{\mathrm{a}}$    \\
 \cline{1-2} 
$R^{1}, R^{2}$ & $(i_{1},j_{1},i_{2},j_{2},k_{2} \mapsto 1, j_{1}-i_{1}, j_{2}-i_{2}, i_{1}, i_{2}, k_{2}, 0)^{\mathrm{a}}$   \\
\cline{1-2} 
$R^{3}, R^{4}$ & $(i_{1},j_{1},i_{2},j_{2},k_{1} \mapsto 1, j_{1}-i_{1}, j_{2}-i_{2}, i_{1}, i_{2}, k_{1}, 0)^{\mathrm{a}}$ \\
 \cline{1-2} 
\hline
\multicolumn{2}{l}{$^{\mathrm{a}}$Parallel Dimension 3 (1-based)}
\end{tabular}
\label{tab:bpm_original_schedule}
\end{center}
\end{table}
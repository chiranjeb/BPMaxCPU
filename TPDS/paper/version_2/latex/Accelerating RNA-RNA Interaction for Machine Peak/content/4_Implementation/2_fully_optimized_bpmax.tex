\subsection{Previous BPMax Schedules}
\textbf{Original BPMax Schedule:}
Let us recall that the original BPMax computed a four-dimensional table $F$-table based on five reductions ($R^{0}$, $R^{1}$, $R^{2}$, $R^{3}$, $R^{4}$) and two two-dimensional tables - $S^{(1)}$ and $S^{(2)}$. \alphaz\ treats each of these entities as a unique variable and requires the user to specify a schedule and a memory map. We have observed previously that $S^{(1)}$ and $S^{(2)}$ require only the input sequences. Thus, they can be scheduled before any other reductions. The schedules of the remaining variables are formulated based on the wavefront parallelization of the 6-D schedule space. We highlight the original program schedule in Table~\ref{tab:bpm_original_schedule}.
\begin{table}[htbp]
\caption{\uppercase{BPMax original Parallelization}}
\begin{center}
\begin{tabular}{|c|c|}
\hline
\textbf{\textit{Reduction}}& \textbf{\textit{Schedules}} \\
\hline
$S^{(1)}$ & $(i_{1},j_{1}, k_{1} \mapsto  0, j_{1}-i_{1}, i_{1}, k_{1}, 0, 0, 0)$   \\
\cline{1-2} 
$S^{(2)}$ & $(i_{2},j_{2}, k_{2} \mapsto  0, j_{2}-i_{2}, i_{2}, k_{2}, 1, 1, 1)$   \\
\cline{1-2} 
$R^{0}$ & $(i_{1},j_{1},i_{2},j_{2},k_{1},k_{2} \mapsto 1, j_{1}-i_{1}, j_{2}-i_{2}, i_{1}, i_{2}, k_{1}, k_{2})^{\mathrm{a}}$    \\
 \cline{1-2} 
$R^{1}, R^{2}$ & $(i_{1},j_{1},i_{2},j_{2},k_{2} \mapsto 1, j_{1}-i_{1}, j_{2}-i_{2}, i_{1}, i_{2}, k_{2}, 0)^{\mathrm{a}}$   \\
\cline{1-2} 
$R^{3}, R^{4}$ & $(i_{1},j_{1},i_{2},j_{2},k_{1} \mapsto 1, j_{1}-i_{1}, j_{2}-i_{2}, i_{1}, i_{2}, k_{1}, 0)^{\mathrm{a}}$ \\
 \cline{1-2} 
\hline
\multicolumn{2}{l}{$^{\mathrm{a}}$Parallel Dimension 3 (1-based)}
\end{tabular}
\label{tab:bpm_original_schedule}
\end{center}
\end{table}

We also recall the most optimized schedule from our prior work ~\cite{Mondal2021} shown in Table~\ref{tab:hybrid_schedule_with_tiling}.


\begin{table}[htbp]
\caption{\uppercase{BPMax hybrid schedule with tiling}}
\begin{center}
\begin{tabular}{|c|c|c|}
\hline
\textbf{} & \textbf{\textit{Variable}}& \textbf{\textit{Schedule}} \\
\hline
 &  Output & $(i_{1},j_{1} \mapsto M, i_{1}, j_{1}, 0)$ \\
 \cline{2-3} 
${\mathrm{a}}$ & $R_{0}$ & $(i_{1},j_{1},k_{1},k_{2} \mapsto k_{1}, i_{1}, k_{2}, j_{1})$,    \\
\cline{2-3} 
& $R_{3}, R_{4}$ & $(i_{1},j_{1},k_{1} \mapsto k_{1}, i_{1}, i_{1}, j_{1})$    \\
% \cline{2-3} 
\hline
%\cline{2-3} 
& a & $(i_{1},j_{1} \mapsto 1, j_{1}-i_{1}, i_{1}, j_{1}-4, 0, 0, 0)$   \\
\cline{2-3} 
${\mathrm{b}}$ & $F$ & $(i_{1},j_{1},i_{2},j_{2} \mapsto 1, j_{1}-i_{1}, M, i_{1}, -i_{2}, j_{2}, 0)$   \\
\cline{2-3} 
& $R_{1}, R_{2}$ & $(i_{1},j_{1},i_{2},j_{2},k_{2} \mapsto 1, j_{1}-i_{1}, M, i_{1}, -i_{2}, k_{2}, j_{2})$  \\
% & & $(i_{1},j_{1},i_{2},j_{2} \mapsto 1, j_{1}-i_{1}, M, i_{1}, -i_{2}, i_{2}-1, j_{2})$ \\
\hline
%\multicolumn{3}{l}{}\\
\multicolumn{3}{l}{${\mathrm{a}}$ - Subsystem schedule(parallel dimension 1)}\\
\multicolumn{3}{l}{${\mathrm{b}}$ - Root system schedule(parallel dimension 3)}\\
\end{tabular}
\label{tab:hybrid_schedule_with_tiling}
\end{center}
\end{table}


\subsection{New BPMax Schedules}
We used the BPMax equation as specified in the original paper ~\cite{EbrahimpourBoroojeny2021} in our previous optimization work ~\cite{Mondal2021}. In our current optimization work, we manually transform the original BPMax equations to apply the second and third-level tiling based on a mono-parametric tile parameter and develop a modified \alfa program. It allows precise scheduling, memory mapping, and transformations of the second-level tiles. We express various computations using subsystems and apply optimization on each of them independently to produce modular code.

Similar to the prior optimization work, we compute the $S^{(1)}$ and $S^{(2)}$ table first. However, we apply mono-parametric tiling on $S^{(2)}$ so that we can use a second-level tile as one of the matrix max-plus operands for computing $R^{1}$ and  $R^{2}$. Let us now discuss the $F$-table schedules at the different tiling levels.

\textbf{First-level Tile Schedule:}
Each first-level tile (an inner triangle) goes through different computation phases (point-wise operations and reductions) associated with a unique subsystem. The point-wise operations initialize (subsystem - $Initialization$) the $F$-table and update (subsystem - $Point\-wise^{sw}$) using the inner triangle to its south-west. Subsystem - $Reductions^{outer}$ accumulates the result from the outer reductions - $R^{0}$, $R^{3}$, $R^{4}$ and subsystem - $Reductions^{inner}$ makes the final update to each tile using the inner reductions - $R^{1}$, $R^{2}$. We process diagonal tiles one at a time for invoking point-wise operations ($Initialization$ and $Point\-wise^{sw}$) and outer reductions $Reductions^{outer}$. $Reductions^{outer}$ accumulates the results from a set of $F(i_{1}, k_{1})$ and $F(k_{1}+1, j_{1})$ triangles where $i_{1} \le k_{1} < j_{1}$.  $2^{nd}$ and $3^{rd}$ dimensions of these subsystem's schedules control the diagonal processing order. \begin{table}[htbp]
\caption{\uppercase{First Level Tile Schedule}}
\begin{center}
\begin{tabular}{|c|c|}
\hline
\textbf{\textit{Subsystem}}& \textbf{\textit{Schedule}}$^{\mathrm{a}}$ \\
\hline
$S^{(1)}, S^{(2)}$ & $(i_{1} \mapsto 0, 0, i_{1}, 0, 0, 0, 0)$   \\
\cline{1-2} 
%\hspace{5mm}
$Initialization$ & $(i_{1},j_{1},i_{2},j_{2} \mapsto 2, j_{1}-i_{1}, i_{1}, -1, i_{2}, j_{2},0)$   \\
\cline{1-2} 
%\hspace{5mm}
$Point\-wise^{sw}$ & $(i_{1},j_{1},i_{2},j_{2} \mapsto 2, j_{1}-i_{1}, i_{1}, -1, i_{2}, j_{2},1)$   \\
\cline{1-2}
$Reductions^{outer}$ & $(i_{1},j_{1},k_{1} \mapsto 2, j_{1}-i_{1}, i_{1}, k_{1}, 0, 0, 0)$    \\
\cline{1-2}
$Reductions^{inner}$ & $(i_{1},j_{1},k_{1} \mapsto 2, j_{1}-i_{1}, M, 0, i_{1}, 0, 0)$    \\
 \cline{1-2} 
\hline
%\multicolumn{2}{l}{}\\
\multicolumn{2}{l}{$^{\mathrm{a}}$Parallel dimension 5}
\end{tabular}
\label{tab:tile_l1}
\end{center}
\end{table}
Notice that initialization of the entire $F$-table is costly due to its footprint. So, we schedule $Initialization$ and $Point\-wise^{sw}$ together ($5^{th}$ and $6^{th}$ dimension of the schedule) before scheduling the $Reductions^{outer}$ (ordering is controlled by the $4^{th}$ dimension of the schedule). After completing all the outer reductions, we schedule $Reductions^{inner}$ ($3^{rd}$ dimension greater than $M-1$) for multiple tiles simultaneously. Table~\ref{tab:tile_l1} outlines the schedule for each one of these subsystems along with $S^{(1)}$ and $S^{(2)}$. The parallel dimension 5 indicates that multiple threads are assigned to do point-wise operations on a particular tile, whereas each tile is finalized ($Reductions^{inner}$) by one thread.


\textbf{Second-level Tile Schedule:} Second-level tiles are processed by $Reductions^{outer}$ and $Reductions^{inner}$ subsystems. $Reductions^{outer}$ accumulates partial results for all the second-level tiles $\{ i_{2}, j_{2} \mid 0 \le i_{2} \le j_{2} \le N_{sec}-1 \}$ for a given first-level tile $F(i_{1}, j_{1})$. It is responsible for scheduling $R^{3}$ (subsystem - $R^{3}_{t}$), $R^{4}$ (subsystem - $R^{4}_{t}$), multiple matrix max-plus input transformations (subsystem - $MT(F)^{A}$ and $MT(F)^{B}$), and multiple matrix max-plus computations (subsystem - $MMP$). For matrix max-plus operation, each second-level tile $F(i_{1}, j_{1}, i_{2}, j_{2})$ is updated using a set of input tiles $F(i_{1}, k_{1}, i_{2}, k_{2})$ and $F(k_{1}+1, j_{1}, k_{2}, j_{2})$ where $i_{2} \le k_{2} \le j_{2}$. As noted earlier, instead of evaluating one output tile at a time, we use a schedule that accumulates results in the output tile. Since $R^{0}$, $R^{3}$, $R^{4}$ share the input tiles, we first schedule the $R^{3}_{t}$ and $R^{4}_{t}$ and reuse the input tiles in $MMP$. We schedule $MT(F)^{A}$ to transform an input tile and use it multiple times as the first operand in a $MMP$ invocation. Before each $MMP$ invocation, we schedule $MT(F)^{B}$ to transform the second operand of the matrix max-plus operation. Table~\ref{tab:bpm_l2_outer_reduction_schedule} highlights the schedule for the different subsystems invoked from $Reductions^{outer}$.
\begin{table}[htbp]
\caption{\uppercase{Second Level Tile, Outer Reductions}}
\begin{center}
\begin{tabular}{|c|c|c|}
\hline
&\textbf{\textit{Subsystem}}& \textbf{\textit{Schedule}}$^{\mathrm{b}}$ \\
\hline
&$R^{3}_{t}$, $R^{4}_{t}$ & $(i_{2},j_{2} \mapsto 0, i_{2},j_{2}, 0, 0, 0)$   \\
\cline{1-3} 
 &$MT(F)^{A}$  & $(i_{2},j_{2} \mapsto 0, i_{2},j_{2}, 0, 0, 1)$   \\
\cline{2-3} 
${\mathrm{a}}$ &$MT(F)^{B}$  & $(i_{2},j_{2},k \mapsto 0, i_{2}, k_{2}, 1, j_{2}, 0)$   \\
\cline{2-3} 
&$MMP$ ($R^{0}$) & $(i_{2},j_{2},k_{2} \mapsto 0, i_{2}, k_{2}, 1, j_{2}, 1)$   \\
\hline
\multicolumn{3}{l}{${\mathrm{a}}$ - Optimized $R_{0}$}\\
\multicolumn{3}{l}{${\mathrm{b}}$ - Parallel dimension 5}
\end{tabular}
\label{tab:bpm_l2_outer_reduction_schedule}
\end{center}
\end{table}

 
\begin{table}[htbp]
\caption{\uppercase{Second Level Tile, Inner Reductions}}
\begin{center}
\begin{tabular}{|c|c|c|}
\hline
\textbf{} & \textbf{\textit{Subsystem}}& \textbf{\textit{Schedule}} \\
\hline
 & $Diagonal Tile$ & $(i_{2},j_{2} \mapsto -i_{2}, j_{2}, 0, j_{2}, 0)$   \\
\cline{2-3} 
\hline
%\cline{2-3} 
 & $MT(S^{2})^{A}$ & $(i_{2},j_{2} \mapsto -i_{2}, j_{2}, 0, j_{2}, 1)$   \\
\cline{2-3} 
${\mathrm{a}}$ & $MT(F)^{B}$ & $(i_{2},j_{2},k_{2} \mapsto -i_{2}, k_{2}, 3, j_{2}, 0)$    \\
 \cline{2-3} 
 &  $MMP$ ($R_{1}$) & $(i_{2},j_{2},k_{2} \mapsto -i_{2}, k_{2}, 3, j_{2}, 1)$ \\
\hline
 & $MT(F)^{A}$ & $(i_{2},j_{2} \mapsto -i_{2}, j_{2}, 4, j_{2}, 0)$    \\
\cline{2-3} 
${\mathrm{b}}$ & $MT(S^{2})^{B}$ & $(i_{2},j_{2},k_{2} \mapsto -i_{2}, k_{2}, 5, j_{2}, 0)$    \\
 \cline{2-3} 
 & $MMP$ ($R_{2}$) & $(i_{2},j_{2},k_{2} \mapsto -i_{2}, k_{2}, 5, j_{2}, 1)$ \\
\hline
& $Finalize$ & $(i_{2},j_{2} \mapsto -i_{2}, j_{2}, 0, j_{2}, 0)$ \\
\hline
\multicolumn{3}{l}{${\mathrm{a}}$ - Optimized $R_{1}$}\\
\multicolumn{3}{l}{${\mathrm{b}}$ - Optimized $R_{2}$}\\
\end{tabular}
\label{tab:bpm_l2_inner_reduction_schedule}
\end{center}
\end{table}


$Reductions^{inner}$ subsystem takes a first-level tile $F(i_{1}, j_{1})$ and $S_{2}$ as input and makes the final update to the $F(i_{1}, j_{1})$. It is responsible for scheduling a diagonal tile (subsystem - $DiagonalTile$), optimized $R_{1}$ (subsystem - $MT(S^{2})^{A}$, $MT(F)^{B}$, $MMP$), optimized $R_{1}$ ($MT(F)^{A}$, $MT(S^{2})^{B}$, $MMP$) and residual patch up computation ($Finalize$). Table~\ref{tab:bpm_l2_outer_reduction_schedule} highlights a bottom-up and left-to-right schedule for these subsystems. So, we first schedule the $DiagonalTile$ tile corresponding to each tile row. For each non-diagonal tile, we schedule all the subsystems that optimize the $R^{1}$, followed by $R^{2}$. Scheduling these subsystems is similar to optimizing $R^{0}$. Finally, we schedule $Finalize$ for each non-diagonal tile to resolve the intra-tile dependencies.

\begin{table*}[htbp]
\caption{\uppercase{Register allocation Strategy}}
\label{tab:ymm_registers}
\begin{center}
\begin{tabular}{|c|c|c|c|c|c|c|c|c|}
\hline
Number & Number & Number of& Number of & Number of YMMs & Total &Number of & Number of   \\
of A & of B & YMMs for A & YMMs for  B&  for accumulations&YMMs Usage &Memory access&  Max-plus  Operations(v)\\
\hline
2 & 24 & 2 & 3 & 6 & 11  & 5 & 6 \\
\hline
2  & 32 & 2 & 4 & 8 & 14 & 6 & 8  \\
\hline
3 & 24 & 3 & 3 & 9 & 15  & 6 & 9 \\
\hline
3 & 16 & 3 & 2 & 6 & 11 & 5 & 6  \\
\hline
4 & 16 & 4 & 2 & 8 & 14  & 6 & 8  \\
\hline
\end{tabular}
\end{center}
\end{table*}

\textbf{Third-level Tile Schedule:}
 We implement the third-level tile using an optimized hand-written register-tiled kernel that performs matrix max-plus. We process the third-level register tile in a column-major order ($i_{3}, j_{3} \mapsto {j_3}, i_{3}$).  The design of the register-tiled kernel is target-dependent. We use Intel intrinsic APIs to compute multiple max-plus operations using Intel Advanced Vector Extensions (AVX-256) registers.

Due to many architectural similarities, we use a common register-tiling implementation for our target architectures - Broadwell and Coffee Lake architecture. One of the main differences between these two architectures is the number of available floating-point addition units (FPA) per core. Even though both architectures have two floating-point multiply-add (FMA) units, Broadwell has only one floating-point add unit, whereas Coffee Lake has two floating-point add units. Since the max operation is also executed using the FPA unit and the number of instructions per cycle for vaddps and vmaxps are twice smaller for Broadwell than Coffee Lake,  Broadwell architecture is significantly bottle-necked for the max-plus computation. The objective of the register tiling is to load the data into the registers and perform as many operations as we can without accessing memory. AVX-256 has sixteen 256-bit registers YMM0-YMM15, which perform a single instruction on multiple data elements. Each YMM register can hold eight single-precision floating points and be used to store operands or results to perform eight single-precision operations. The execution latency of vaddps and vmaxps operation on Coffee Lake is four cycles (3 for the Broadwell). Thus, we need to have 8 (6 for the Broadwell) independent chains of computations to fully utilize both FPA execution ports for Coffee Lake.

We are interested in a data access pattern of $C = (a + B) \max C $, where $a$ is a scalar and $B$, $C$ are vectors. So, we load eight consecutive elements (vector) of $B$ and $C$ into the YMM registers (B, C) but load a single element (scalar) of $a$ and broadcast it to a YMM register(A). The goal is to find the combination of $A$ and $B$ that maximizes CPU-resource utilization. Table~\ref{tab:ymm_registers} shows the different register tiles ($ A\times B$) that maximize the resource utilization but minimize the AVX register allocation. We notice that $3 \times 24$ maximizes the resource utilization with the best memory access to compute ratio. 

\textbf{Memory Access:} We implement several techniques to optimize memory access. Each core is responsible for executing a complete matrix max-plus operation that requires data transformation. We use a dedicated buffer for these transformations and select them using omp\_get\_thread\_num(). We ensure the processor-to-memory affinity by setting OMP\_PLACES to 'cores' and set OMP\_PROC\_BIND to true to bind the OMP threads to the physical core to improve data locality during the on-the-fly memory transformation. We use Intel intrinsic for allocating these transformation buffers so that they are aligned to SIMD width ($8 \times 4$ bytes).




